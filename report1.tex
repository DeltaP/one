\documentclass[11pt,a4paper,oneside]{report}

\begin{document}
\title{HPSC Assignment 1}
\author{Gregory Petropoulos}
\date{September 12, 2012}
\maketitle

\section{Tree reduce pseudocode}

I chose to do problem 1.3 and wrote a short program using perl.  Please see the attached code, below is output generated for 16 cores.
\\\\
\noindent\texttt{* * * *\\}
\noindent\texttt{Process 0 receiving from process 1 and performing local sum\\
Process 1 sending to process 0\\
Process 2 receiving from process 3 and performing local sum\\
Process 3 sending to process 2\\
Process 4 receiving from process 5 and performing local sum\\
Process 5 sending to process 4\\
Process 6 receiving from process 7 and performing local sum\\
Process 7 sending to process 6\\
Process 8 receiving from process 9 and performing local sum\\
Process 9 sending to process 8\\
Process 10 receiving from process 11 and performing local sum\\
Process 11 sending to process 10\\
Process 12 receiving from process 13 and performing local sum\\
Process 13 sending to process 12\\
Process 14 receiving from process 15 and performing local sum\\
Process 15 sending to process 14\\}

\noindent\texttt{Process 0 receiving from process 2 and performing local sum\\
Process 2 sending to process 0\\
Process 4 receiving from process 6 and performing local sum\\
Process 6 sending to process 4\\
Process 8 receiving from process 10 and performing local sum\\
Process 10 sending to process 8\\
Process 12 receiving from process 14 and performing local sum\\
Process 14 sending to process 12\\}

\noindent\texttt{Process 0 receiving from process 4 and performing local sum\\
Process 4 sending to process 0\\
Process 8 receiving from process 12 and performing local sum\\
Process 12 sending to process 8\\}

\noindent\texttt{Process 0 receiving from process 8 and performing local sum\\
Process 8 sending to process 0\\}
\noindent\texttt{* * * *\\}

\section{Ring "Hello World"}

This program passes a message to process $ (i+1) \%p $ and wraps around such that the highest process sends a message to process zero.  I tested this program by running with a variety of process numbers including running with only one process.  I believe the implementation is correct because every process sends a message to the correct process and receives from the correct process.  The program should send first and then receive because otherwise all the processors will be waiting for a message that never comes and the program will hang.  A better implementation that reduces the load on the MPI buffer is to have half of the processes send and the other half receive first.  This is how I implemented the message passing.  When the program is run on one processor I have an if statement that catches this case and outputs a message saying only one processor is being used so there is no one to pass the message to.  Since there was no mention of ordering the output in any particular way I allowed the output to be non deterministic.  Output is reported such that, if the program is run on two cores, the second core will print a greeting from the first core like so:  \texttt{Greetings from process 0 of 2 | reported by process 1 of 2}.\\

Output for 1 process:

\noindent{\texttt{No need to pass messages; only one process.\\}

Output for 4 processes:

\noindent{\texttt{Greetings from process 3 of 4 | reported by process 0 of 4\\
Greetings from process 2 of 4 | reported by process 3 of 4\\
Greetings from process 0 of 4 | reported by process 1 of 4\\
Greetings from process 1 of 4 | reported by process 2 of 4\\}

Output for 5 processes:

\noindent\texttt{Greetings from process 0 of 5 | reported by process 1 of 5\\
Greetings from process 2 of 5 | reported by process 3 of 5\\
Greetings from process 1 of 5 | reported by process 2 of 5\\
Greetings from process 4 of 5 | reported by process 0 of 5\\
Greetings from process 3 of 5 | reported by process 4 of 5\\}

\section{Simpson's Rule}

This program calculates the integral of a function numerically over an integral [a,b] using Simpson's Rule.  I read in a, b, n, and possibly a -verbose flag from the command line.  If too few, too many, or the wrong flag is provided the program will terminate and print an appropriate error message.  The program will also terminate, printing an error message, if $(p \% n) !=0$.

I tested this program by calculating a variety of functions over a variety of intervals [a,b] while varying the number of cores and the number of integration intervals.  I compared the results of my parallel program with my serial code as well as my TI-89 calculators numerical integrate function, in all cases the three methods give the same numeric result.\\

Results for the verbose option when integrating $f(x)=x$ from 0 to 5 using 4 processors and 16 intervals.\\

\noindent\texttt{Process 1:  2 4 2 4 \\
Process 2:  2 4 2 4 \\
Process 3:  2 4 2 4 1 \\
Process 0:  1 4 2 4 \\
The integral is: 12.5\\
Program Complete \\}

Results for integrating $f(x)=2*x^5-x^4+5*x^3-10*x^2+15*x-3$ on gordon from 0 to 64 using 64 processors and 6400 intervals.\\

\noindent\texttt{The integral is: 2.27119e+10\\
Program Complete\\
Nodes:        gcn-9-31 gcn-9-32 gcn-9-33 gcn-9-37\\
Out/simpson.out (END) \\}

\end{document}
